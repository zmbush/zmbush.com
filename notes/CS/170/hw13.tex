% Search for all the places that say "PUT SOMETHING HERE".

\documentclass[11pt]{article}
\usepackage{amsmath,textcomp,amssymb,geometry,graphicx}

\def\Name{Zachary Bush}  % Your name
\def\Sec{Th 4-5; Starfield}  % Your discussion section
\def\Login{cs170-nx} % Your login
\def\HW{13}

\title{CS170--Spring 2012 --- Solutions to Homework \HW}
\author{\Name, section \Sec, \texttt{\Login}}
\markboth{CS170--Spring 2012 Homework \HW \Name, section \Sec}{CS170--Spring 2012
Homework \HW \Name, section \Sec, \texttt{\Login}}
\pagestyle{myheadings}

\begin{document}
\maketitle

\section*{1.}
Show that Dominating Set problem is NP-Complete.

To show that this problem is NP-Complete, we will reduce a known NP-Complete
problem to this one and thereby show that this problem is at least as hard as
the other. 

We will in this case be reducing Set Cover to dominating set.

The reduction is as follows:

For each subset in the set cover problem we will create a node $s_n$. We turn
all of these $s$ variables into a Clicque so that selecting any set will touch
all other sets. 

From each of these sets, we create edges to each element $e_n$ in that subset. 

For example, if set $s_k$ had elements $\left\{ e_1, e_3, e_4 \right\}$, then
there would be edges from the node $s_k$ to $e_1$, $e_3$, and $e_4$.

From here we simply find the dominating set of this resulting graph. 

This is almost the answer. 

We have to perform a post-process that will look through each of the $e_n$
elements. If any of these elements have been selected, then we instead select
any of its connected sets. 

We thereby add no new nodes to the dominating set, and our final answer will be
the sets that were selected by the dominating set problem.


The logic of this algorithm is as follows. Each set selects a certain number of
elements, if each set node is connected to each of its element, then selecting
the set in dominating set is the same as selecting the set in set cover. We
have all the sets connected together to ensure that a selection of at least one
set will be the answer. There are certain cases where either an element or a
set could be selected, such as the case of a single set with a single element.
In that case, we use our post-process to enforce that sets are the only things
selected. 

If we had an optimal dominating set where an element is selected, then
switching that selection for one of its covering sets will not reduce the
optimality. Furthermore, if an element with no covering set is selected, then
we can signal a failure as there is no solution to the initial set cover
problem.
\newpage
\section*{2.}
To show that connected dominating set is NP-Complete, we can perform the exact
same reduction as we did in part 1, since all of the sets are connected to each
other, a solution to the first part is actually going to be a connected
dominating set. 
\end{document}
