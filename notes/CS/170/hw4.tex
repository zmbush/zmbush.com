% Search for all the places that say "PUT SOMETHING HERE".

\documentclass[11pt]{article}
\usepackage{amsmath,textcomp,amssymb,geometry,graphicx}

\def\Name{Zachary Bush}  % Your name
\def\Sec{Th 4-5; Starfield}  % Your discussion section
\def\Login{cs170-nx} % Your login

\title{CS170--Spring 2012 --- Solutions to Homework 4}
\author{\Name, section \Sec, \texttt{\Login}}
\markboth{CS170--Spring 2012 Homework 4 \Name, section \Sec}{CS170--Spring 2012 Homework 3 \Name, section \Sec, \texttt{\Login}}
\pagestyle{myheadings}

\begin{document}
\maketitle

\section*{1.}

\begin{itemize}
\item[(i)]
\includegraphics[scale=.50]{1-1.pdf}
\begin{itemize}
\item[(a)]
The order is: GHI, A, B, E, CDFJ
\item[(b)]
C,D,F,J is a sink SCC

B and E are Source SCCs
\newpage
\item[(c)]
\includegraphics[scale=.50]{1-1-1.pdf}
\item[(d)]
Two. Connect $F \rightarrow E$ and $E\rightarrow B$
\end{itemize}
\item[(ii)]
\includegraphics[scale=.50]{1-2.pdf}
\begin{itemize}
\item[(a)]
FIHGD, then C, and then ABE
\item[(b)]
ABE is a source

FIHGD is a sink
\item[(c)]
\includegraphics[scale=.50]{1-2-1.pdf}
\item[(d)]
One edge from FIGHD to ABE
\end{itemize}
\end{itemize}
\newpage
\section*{2.}
The method to reverse the adjacency list is very simple. First we create an
empty adjacency list. Next, we go through each node, and iterate over its
neighbors. We then look up this node in the hashtable of nodes, and add at the
beginning of its adjacency list the node we are considering. Thus, we reverse
the graph in linear time $O(|V|)$
\newpage
\section*{3.}
Finding the number of semesters it would take to take all the required classes,
is very similar to finding the longest possible path through a DAG. Since there
are no cycles, the problem is very simple. 

First, we take the DAG, invert it, and sort it topographically, which can be
done in linear time. Next, we go through this topographically sorted list, and
for each element, set it's value to the maximum value of all of its parents plus
one. In this way, once we reach the end, we take the largest value of any node
to be the number of semesters required to take all the classes. 

\newpage
\section*{4.}
If we think of this problem in terms of strongly connected components, then this
problem can be solved very simply. The first thing we can always do is separate
the graph into all of its strongly connected components in linear time. 

On each SCC, we can perform the following operation. 

Start at any node. Assign it one of two colors Blue, and Gold
As we traverse through the graph, if we come from a blue node, we color the
current one gold and vice versa. After we finish this operation (which is
$O(|N|)$), we look through each edge in the SCC, and if an edge connects two
elements of the same color, then we have an odd cycle. From there we can
linearly traverse the SCC until we come back to the other node, and we have the
odd cycle. 
\newpage
\section*{5.}
In order to determine if there is a path that touches every node, we can simply
sort the DAG topographically, For each element, it must have an arc that
connects it to its direct predecessor in the list. If this is not the case, then
there does not exist a path that touches every node. 
\newpage

\section*{6.}

\begin{itemize}
\item[(a)]
\begin{eqnarray*}
(true, false, false, true)\\
(true, true, false, true)
\end{eqnarray*}
\item[(b)]
A very simple example is:
\begin{equation*}
(x_1 \lor x_2) \land (x_1 \lor \lnot x_2) \land (\lnot x_1 \lor x_2) \land
(\lnot x_1 \lor \lnot x_2) \land (x_3 \lor x_4)
\end{equation*}
\item[(c)]
\includegraphics[scale=0.5]{6-1.pdf}
\includegraphics[scale=0.5]{6-2.pdf}
\item[(d)]
If there is a strongly connected component with both a value and its negation,
then if you were to assign True to one of them, we would have to follow the
edges and assign each value True in sequence, until you reach its negation, and
you would produce a contradiction. 
\item[(e)]
Using this method, we never reach a point where we would cause an assignment to
not work. 
Since anything implies true is always true, and false implies anything is always
true. In this way, we assign values safely until we have assigned all required
values. 
\item[(f)]
In order to separate the graph into SCC, we can perform this operation in linear
time. The assignment to each sink SCC is linear time, and if ever an SCC
contains both a value and its negation, then we know there is no safe
assignment. Thus we can solve a 2-SAT in linear time. 
\end{itemize}


\end{document}
