% Search for all the places that say "PUT SOMETHING HERE".

\documentclass[11pt]{article}
\usepackage{amsmath,textcomp,amssymb,geometry,graphicx}

\def\Name{Zachary Bush}  % Your name
\def\Sec{Th 4-5; Starfield}  % Your discussion section
\def\Login{cs170-nx} % Your login
\def\HW{10}

\title{CS170--Spring 2012 --- Solutions to Homework \HW}
\author{\Name, section \Sec, \texttt{\Login}}
\markboth{CS170--Spring 2012 Homework \HW \Name, section \Sec}{CS170--Spring 2012
Homework \HW \Name, section \Sec, \texttt{\Login}}
\pagestyle{myheadings}

\begin{document}
\maketitle

\section*{1.}
Give an example of a quadratic programming problem such that the feasible
region is nonempty and bounded, and yet none of the vertices maximize your
function. 

\begin{eqnarray*}
\min (x_1-1)^2 \\
x_1, x_2 \ge 0\\
x_2 \le 2\\
x_1 \le 2
\end{eqnarray*}
The optimal: $(1, 0)$, is not at a vertex. 


\newpage
\section*{2.}
If we start from the maximization problem:
\begin{eqnarray*}
\max x_1 - 2x_3\\
x_1 - x_2 \le 1\\
2x_1 - x_3 \le 1\\
x_1, x_2, x_3 \ge 0
\end{eqnarray*}

We can prove the optimality of the solution $(3/2, 1/2, 0)$, by computing the
dual, which is:
\begin{eqnarray*}
\min y_1 + y_2\\
y_1 \ge 1\\
y_2 \le 2\\
2y_2 - y_1 \ge 0
\end{eqnarray*}

The optimal solution for this dual is: $(1, 1/2)$, whose value is: $3/2$, which
is the same value as for the primal problem. Therefore $(3/2, 1/2, 0)$ is indeed
optimal.
\newpage
\section*{3.}
The payoffs for the row player are as follows:
\begin{tabular}[h]{c|c|c|}
&H&T\\
\hline
H&1&-1\\
\hline
T&-1&1\\
\hline
\end{tabular}


The optimal strategies for the two players are for them to play uniformly at
random. 

The value for this game is: 
\begin{equation*}
(1/2)(-1) + (1/2)(1) = 0
\end{equation*}

\newpage
\section*{4.}
From the start node, we connect to each vertex in the graph with an edge of
infinite width. Each vertex then has an outgoing edge with a weight equal to the
number of edges connecting to it. The destination node also has edges backwards
to each vertex that it removes an edge from. Each of these nodes point to the
final node. Thus we construct a bipartate graph that represents our minimum
vertex cover.

From there, all we have to do is find the maximum flow through the system, and
we have our answer. 
\newpage
\section*{5.}
\begin{enumerate}
\item[(a)]
\begin{eqnarray*}
\max f_{at} + f_{bt}\\
f_{sa} \le 1\\
f_{sb} \le 3\\
f_{ab} \le 1\\
f_{at} \le 2\\
f_{bt} \le 1\\
f_{ab} + f_{at} - f_{sa} = 0\\
f_{bt} - f_{ab} - f_{sb} = 0
\end{eqnarray*}

\item[(b)]
The dual for this problem is:
\begin{eqnarray*}
\min y_1 + 3y_2 + y_3 + 2y_4 + y_5\\
y_4 \ge 1\\
y_5 \ge 1\\
y_1,y_2,y_3 \ge 0
\end{eqnarray*}

\item[(c)]
The dual for the general case is as follows:
\begin{eqnarray*}
\min \sum_{(a,b) \in E} c_{ab}y_{ab}\\
\forall (s, n) \in E,\quad y_{sn} \ge 1\\
\forall (a, b) \in E,\quad y_{ab} \ge 0
\end{eqnarray*}

\item[(d)]
Since every edge $y_{sn}$ from the start node, must be greater than or equal to
one, than any path from s, must also have a value greater than or equal to one.
\end{enumerate}
\newpage
\section*{6.}
\begin{enumerate}
\item[(a)] If we always chose to use the flow of 1 from each edge, that is to
say, each time we add a new path, the total flow increases by one, then in order
to find the optimal flow (2000), we would have to run the algorithm over 1000
times. 
\item[(b)] The fattest path can be found with a variant of the Dijkstra
algorithm. Instead of trying to find the minimum value, we try to maximize the
value, and when we update the value of a path, we take the minimum of the
current value and the value of the newly added edge. In this way, we can easily
find the fattest path from the start node to the end node. 
\item[(c)] Each of the paths found by this modified Dijkstra algorithm can be
thought of as maximizing the flow through at least one edge. If we repeat this
search, we will come up with at most $|E|$ different paths each of which
maximize at least one edge. Once an edge is maximized, it needs never be
included in the search again. As such, once we find our set of paths that
maximize flow for each of the edges, the total flow is simply the sum of these
edges. 
\item[(d)] Each time we run this procedure of increasing flow along the fattest
path, we must add at least $\log_{|E|} V$ flow to the current flow. We will have
to run this procedure at most $|E|$ times to get the total number of paths.
Which gives us a final runtime of $O(|E|\log v)$.
\end{enumerate}
\end{document}
